\documentclass[a4paper, 12pt]{article}
\usepackage[utf8]{inputenc}
\usepackage[francais]{babel}
\usepackage[pdftex]{graphicx}
\usepackage{geometry}
\geometry{hmargin=2cm,vmargin=1.5cm}

\begin{document}
\begin{titlepage}
\begin{center}

{\Large Université de Mons}\\[1ex]
{\Large Faculté des sciences}\\[1ex]
{\Large Département d'Informatique}\\[2.5cm]

\newcommand{\HRule}{\rule{\linewidth}{0.3mm}}
% Title
\HRule \\[0.3cm]
{ \LARGE \bfseries Projet Reseau 1 \\[0.3cm]}
{ \LARGE \bfseries Selective Repeat et Congestion  \\[0.1cm]} % Commenter si pas besoin
\HRule \\[1.5cm]

% Author and supervisor
\begin{minipage}[t]{0.45\textwidth}
\begin{flushleft} \large
\emph{Professeur:}\\
Bruno \textsc{Quoitin}\\
Jérémy \textsc{Dubrulle}
\end{flushleft}
\end{minipage}
\begin{minipage}[t]{0.45\textwidth}
\begin{flushright} \large
\emph{Auteur:} \\
Laurent \textsc{Bossart} \\
Guillaume \textsc{Proot} 
\end{flushright}
\end{minipage}\\[2ex]

\vfill

% Bottom of the page
\begin{center}
\begin{tabular}[t]{c c c}
\includegraphics[height=1.5cm]{logoumons.jpg} &
\hspace{0.3cm} &
\includegraphics[height=1.5cm]{logofs.jpg}
\end{tabular}
\end{center}~\\
 
{\large Année académique 2018-2019}

\end{center}
\end{titlepage}

\tableofcontents

\pagebreak

\section{Construction et exécution}
	Pour compiler notre programme, il faut se placer dans le dossier contenant le package reso et entrer la commande suivante : \textit{javac reso/examples/selectiverepeat/*.java -Xlint}.
	Pour exécuter notre programme, il suffit d'entrer la commande suivante: \textit{java reso.examples.selectiverepeat.Demo} en passant en paramètres: un entier qui est le nombre de paquets que l'on veut envoyer, deux floats qui sont respectivement le taux de perte de paquets et de ACKs (flotants compris entre 0 et 1. Où 1 représente une perte de 100 \%). Si un ou plusieurs paramètres sont maquants ou incorrectes, les paramètres par défault seront appliqués, càd respectivement 10, 0.2, 0.0.

\section{Approche utilisée pour l'implémentation}
	\subsection{Selective Repeat}
		\begin{itemize}
			

		\end{itemize}

	\subsection{Congestion Control}
		\begin{itemize}
			
		\end{itemize}

\section{Les difficultés rencontrées}
	
	\subsection{difficultés liées à l'utilisation du simulateur}
		\begin{itemize}

		\end{itemize}

	\subsection{Difficultés liées à l'implémentation de Selective Repeat et de Congestion Control}
		\begin{itemize}
			
		\end{itemize}

\section{L'état de l'implémentation finale}
	

\section{Note d'utilisation}
	\begin{itemize}
		\item Etant donné que pour chaque utilisation la taille de la fenetre ne sera pas constante, lors de l'exécution de notre programme nous stockons dans un fichier \textit{data.txt} les données de la taille de la fenetre de la dernière exécution.
	La colonne de gauche représente le temps du scheduler et la colonne de droite la taille de la fenètre à ce moment là.
	Pour créer le plot de la dernière exécution, il faut renter la commande suivante via gnuplot:
	\begin{center}
		\textit{plot "data.txt" with linespoints}
	\end{center} 
	\item Nous avons choisi d'afficher les logs du déroulement de paquets directement dans la console.
	\end{itemize}
	

\end{document}
